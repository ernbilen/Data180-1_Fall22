\documentclass[11pt,letter]{article}
\usepackage[left=1.3in,right=1.3in,top=.9in,bottom=1in]{geometry}

\usepackage{setspace,mdwlist,comment}
\setlength{\marginparwidth}{.5in}
\setlength{\parindent}{0in}
\setlength{\parskip}{1em}

% mini table of contents
\usepackage{minitoc}
\dosecttoc % make section toc
\setcounter{secttocdepth}{2} % subsection depth
\renewcommand{\stctitle}{} % no title
\nostcpagenumbers


\usepackage{hyperref}
\hypersetup{
     colorlinks=true,
     linkcolor=magenta,
     filecolor=blue,
     citecolor = black,
     urlcolor=magenta,
    pdfnewwindow=true,
    pdfborder={0 0 0}
}


\usepackage{titlesec}
\usepackage{titling}
\usepackage{xcolor}
\usepackage{multirow,booktabs,colortbl,tabularx}
\newcolumntype{C}[1]{>{\centering\arraybackslash}m{#1}}
\newcolumntype{L}[1]{>{\raggedright\arraybackslash}m{#1}}
\usepackage{graphicx}
\usepackage{longtable}



\titleformat{\section}
  {\sffamily\large\bfseries}
  {\thesection}{1em}{}
\titlespacing\section{0pt}{0pt}{0pt}



%bch: charter

\title{{\sffamily\large\textbf{DATA/MATH/COMP 180: Introduction to Data Science}\\ \ttfamily Fall 2022, Section 1}}
\author{}
\date{}



\usepackage{enumitem}
\newcounter{dummy}
\usepackage{mathpazo}

\begin{document}



\faketableofcontents


\maketitle

\vspace{-10em}

%\vspace{-10.4em}

\textit{(Last updated \today{}; syllabus is subject to change)}

%\vspace{-1mm}


\begin{minipage}[t]{0.61\linewidth}
\textbf{Instructor:} Professor Eren Bilen\\
\textbf{Office:} Rector North 1309\\
\textbf{Email:} \href{bilene@dickinson.edu}{\url{bilene@dickinson.edu}} \\
\textbf{Phone}: 717-254-8162\\
\textbf{Office Hours:} Monday 4:30-5:30pm \\ \hspace*{6.5em}Tuesday 3:00-4:00pm \\ \hspace*{6.5em}Wednesday 9:00-10:00am \\ \hspace*{5.8em} or by appointment \\
\textbf{QRA:} Ashley Doan, \href{phaml@dickinson.edu}{\url{doana@dickinson.edu}}\\
\textbf{Office Hours:} Tuesday 6:30:8:30pm\\ \textbf{Location:} Rector North 1311 \\
\end{minipage}
\begin{minipage}[t]{0.4\linewidth}
\textbf{Class:} Tome 120\\\hspace*{3.5em}Tuesday and Thursday \\\hspace*{3.5em}9:00-10:15am  \\
\hspace{1em}\\
\hspace{1em}\\
\end{minipage}

\section*{Class Notes and Other Required Materials}
\begin{itemize}[nosep]
	\item MATH 180: Introduction to Data Science Course Packet by Jeff Forrester, available at the Dickinson College bookstore
  \item Introduction to Statistical Learning by Gareth James, Daniela Witten, Trevor Hastie, Robert Tibshirani (optional)
	\item Reasoning with Data by Jeff Stanton (optional)
	\item Access to a computer to install and use R
	\item Course webpage: \href{https://lms.dickinson.edu}{\url{Github}}
\end{itemize}




\section*{Course Goals}
The ability to work with and derive information from ever-increasing amounts of data will be one of the important stories of the 21st century. New analytical techniques coupled with rapidly advancing computational power continues to change way data is collected, organized, analyzed, and understood. A facility with data science techniques allows the student to bring this exciting new toolkit to bear helping to mine information from almost every area of human interest. DATA 180 provides an introduction to the core ideas of data science. Topics include data visualization, data wrangling, statistical measures of center, spread, and position, and supervised and unsupervised statistical/machine learning. Upon successful completion of the course a student will be able to:
\begin{itemize}[nosep] 
	\item Organize, manipulate, and transform data using R,
	\item Use Github and RMarkdown to create reproducible reports and maintain a repository for version control,
	\item Analyze and interpret data using visualization techniques and statistical summaries,
	\item Employ supervised and unsupervised machine learning techniques for predictive modeling,
	\item Identify internal structure in data organize, manipulate, and transform data in a statistical programming environment, 
	\item Comprehend and create basic numerical and/or logical arguments.
\end{itemize}
%\newgeometry{left=1.4in,right=1.4in,top=1in,bottom=1in}

We will make extensive use of the R and R-Studio to generate graphical and numerical representations of data, and apply basic machine learning techniques while we interpret the results. R is a fun and useful computational tool as well as an immediate resume builder!

\section*{Course Policies}
\textbf{Attendance Policy:} This course will be taught in person in Tome 120. Students are expected to attend all in-class meetings, which occur on Tuesdays and Thursdays from 9:00-10:30am EDT. While I will not take formal attendance, it is important for you to attend the class meetings and take notes. If you will be unable to attend a class meeting for any health-related issues or other emergencies, please contact me beforehand so that arrangements can be made.

\textbf{Use of Laptops, Tablets, and Phones:}  Laptops and tablets are permitted for note-taking during this course. In exchange for trusting you to use these devices, I ask that you not use them as distractions. I maintain the right to change this policy for individual students or for everyone if these tools become a problem during class. Phones are not permitted and should be put away in silent mode.

\textbf{Grading:} Your course grade is based on two closed-book midterms, a take-home final exam, and homework assignments.

%\hspace*{3.4cm}
\begin{minipage}{.65\textwidth}
Midterm 1 (20\%): \\
Midterm 2 (20\%): \\
Take-home Final (20\%): \\
Homework (40\%):
\end{minipage}%
\hspace*{-5cm}
\begin{minipage}{.5\textwidth}
October 13 \\
December 1 \\
by December 12, 2pm EDT \\
Due dates TBA


\end{minipage} \\

While I will not be giving extra credit in this course, I will drop your lowest homework. I expect there to be 10-12 total assignments (depending on course pacing). Occasionally, an assignment may be weighted to count as two assignments (because of the complexity or length), this will be clearly indicated when it is assigned.

 %\clearpage
The following scale will be used to determine your final grade:

\begin{center}
\begin{tabular}{lll@{\hskip .5in}lll}
\textbf{Score} & \textbf{Letter} & \textbf{GPA} & \textbf{Score} & \textbf{Letter} & \textbf{GPA}\\
\hline
$93 \geq x$	&	A	&	4.0 &  $73 \leq x < 77$	&	C	&	2.0\\
$90 \leq x < 93$	&	A-	&	3.7 & $70 \leq x < 73$	&	C-	&	1.7\\
$87 \leq x < 90$	&	A-	&	3.3 & $67 \leq x < 70$	&	D+	&	1.3\\
$83 \leq x < 87$	&	B+	&	3.0 & $63 \leq x < 67$	&	D	&	1.0\\
$80 \leq x < 83$	&	B	&	2.7 & $60 \leq x < 63$	&	D-	&	0.7\\
$77 \leq x < 80$	&	C+	&	2.3 & $x < 60$	&	F	&	0.0\\

\end{tabular}
\end{center}

\textbf{Make-up Exams:} There will be no make-up exams unless a student must be away from campus on university business or due to an emergency. The student must provide documentation. If an emergency arises, you must inform me as soon as possible. Once you provide me an official documentation related to the emergency/university business, you may schedule a make-up exam. Warning! It is absolutely essential to provide me documentation. You will receive 0 if you are unable to get an official documentation. Therefore, you should definitely not skip a test if your situation cannot produce documentation.

\textbf{Homework:} Homework assignments will be posted on course Github page as an R-Markdown file template on which you will insert your solutions. Due dates will be provided for each assignment. You will turn in your assignments as an R-Markdown file via a pull request from your private GitHub.com repository which is a clone of the class master repository. (You will need to set up a GitHub account if you do not already have one.) You will be sent an invitation link for each assignment. After accepting the assignment, your private repo where you will push your files will automatically be created. Prior to pushing your submission files to your repository, make sure to hit \texttt{Knit} on R-Studio, and include the \texttt{.Rmd} file in your commit. Make sure your code executes with no issues. You will receive a 20\% penalty if any part of your code cannot get executed because of errors. \underline{Email submissions will not be accepted.} Late assignments will not be graded.

You are encouraged to work in teams, but your submissions must be individual. It is important that you must understand and be able to explain every part of the code you are submitting. I do not want to see a bunch of copies of identical code. I do want to see each of you learning how to code these problems so that you could do it on your own. Homework assignments will require the use of R and R-Studio; you will want to obtain access to a computer with R-Studio installed during the first week of classes; R is installed in Tome 121 and various labs in Tome Hall.

\textbf{Take-home Final:} The course will include a final written data science assessment in lieu of a final exam that will be due Wednesday, December 12 at 2:00 pm EDT. Similar to many take-home data scientist interviews, you will have a fixed duration e.g., 24-48 hours to prepare your analysis. You are allowed to refer to your notes, or any online resources, help files, docs. More information will be posted later in the semester.

\section*{Getting Help}

\textbf{Office Hours:} I will be holding three hours of office hours each week.  Please see the first of page of the syllabus for my hours. I am also available by appointment. If there is a conflict and you are unable to make it to any of my hours, please feel free to send me an email. My availability outside office hours is not guaranteed, however I devote my attention fully to you during my office hours. Therefore, I highly encourage you to come to my office hours and ask questions.

\textbf{Quantitative Reasoning Associate:} This semester, we are fortunate to have a Quantitative Reasoning Associate (QRA) working with us. A
QRA is a fellow student who completed this course in the past and will be helping us as a course facilitator and student mentor. This semester, the QRA for our course is Ashley Doan. She will be holding office hours as posted on the first page of the syllabus. \par
Additionally, Ashley will host study sessions before each exam, which will be announced closer to exams.


\section*{Quantitative Reasoning Center}

Dickinson College provides additional support for students taking courses with quantitative content across the curriculum through the Quantitative Reasoning (QR) Center. For the fall 2021 semester, the QR Center will offer tutoring for DATA 180, in addition to general quantitative support. You are strongly encouraged to make an appointment with them. \href{https://www.dickinson.edu/info/20525/quantitative_reasoning_center/2962/quantitative_reasoning_center}{\texttt{Click here}} to access the QR Center webpage.

Please visit \href{https://dickinson.mywconline.com}{\texttt{dickinson.mywconline.com}} to make an appointment. Then, access the drop-down menu under ``limit to" at the top of the scheduler and select DATA 180. This will restrict the tutor list and schedule to only those tutors approved for this course. When you make your appointment, please also paste or upload your assignment and any work that you have done.


\section*{Other Important Information}

\textbf{Referencing the Work of Others:}  When submitting your work, you must follow common-sense ground rules.  External sources may only be used to improve your own understanding of the material.  When you write your solutions, you should do it on your own without the direct help of any external sources, and certainly should not write down anything that you do not understand. If you do use external references, please be sure to cite them.  Failure to cite references will be treated as academic dishonesty.

\textbf{Respect for Intellectual Property:} It is important that you be aware of and respect the intellectual property rights of others. Unless explicitly stated otherwise, all materials available on the Internet, in libraries, and elsewhere are considered intellectual property and can only be used with the permission of the owner. Specifically, with regards to this class, you should not share any of the course materials, including homework answer keys, with others, even after the completion of the course.

\textbf{Statement on Disabilities:} Dickinson values diverse types of learners and is committed to ensuring that each student is afforded equitable access to participate in all learning experiences. If you have (or think you may have) a learning difference or a disability -- including a mental health, medical, or physical impairment – that would hinder your access to learning or demonstrating knowledge in this class, please contact Access and Disability Services (ADS). They will confidentially explain the accommodation request process and the type of documentation that Dean and Director Marni Jones will need to determine your eligibility for reasonable accommodations. To learn more about available supports, go to \url{www.dickinson.edu/ADS}, email \url{access@dickinson.edu}, call (717) 245-1734, or go to the ADS office in Room 005 of Old West, Lower Level (aka "the OWLL").

If you have already been granted accommodations at Dickinson, please follow the guidance at \url{www.dickinson.edu/AccessPlan} for disclosing the accommodations for which you are eligible and scheduling a meeting with me as soon as possible so that we can discuss your accommodations and finalize your Access Plan. If test proctoring will be needed from ADS, remember that we will need to complete your Access Plan in time to give them at least one week’s advance notice.

\textbf{SOAR: Academic Success Support}: Students can find a wealth of strategic guidance by going to \url{www.dickinson.edu/SOAR}. This website for SOAR (Strategies, Organization, and Achievement Resources) includes apps, tips, and other resources related to time management, study skills, memory strategies, note-taking, test-taking, and more. You will also find information aimed to helps students ``SOAR Through Academic Challenges," as well as a schedule of academic success workshops offered through Academic Advising. If you would like to request one-on-one assistance with developing a strategy for a manageable and academically successful semester, email \url{SOAR@dickinson.edu}.

\textbf{Course Outline:} Below is a list of topics to be covered in this course. There may be adjustments on the list during the semester depending on progress. Any adjustments will be announced and this syllabus will be updated. \vspace{-2mm}
\begin{itemize}
	\item Topic 1: Introduction to Data Science, R intro
	\item Topic 2: Data and Variables
	\item Topic 3: Visualization in R
	\item Topic 4: Introduction to Data Wrangling: Tidyverse
	\item Topic 5: Unsupervised Learning: Cluster Analysis
  \item Topic 6: Introduction to Text Mining
	\item Topic 7: Introduction to Supervised (Machine) Learning
\end{itemize}

\vspace{.2cm}

\hspace{-15mm}
\scalebox{.83}{
\begin{tabular}{ll}
\textbf{Important Dates for the Fall 2022 Semester} & \\
\hline
Last Day to Add/Drop or Change to/from Pass/Fail &	Friday, September 2 \\
Mid-Term Pause	&5 pm, Friday, October 14 thru 8 AM, Wednesday, October 19 \\
Course Request Period for Spring 2022 Semester	&Monday, October 31 thru Wednesday, November 2 \\
Thanksgiving Vacation&	5PM, Tuesday, November 22 thru 8 AM, Monday, November 28 \\
Last Day to Withdraw from a Course with a "W" grade	& Tuesday, November 22 \\
Classes End	&Friday, December 9 \\
Reading Period Days &	December 10, 11 \\
\hline
\end{tabular}} \refstepcounter{dummy}

\clearpage

\iffalse
\clearpage
\newgeometry{left=1in,right=.1in,top=.5in,bottom=.1in}
\vspace*{-.1cm}
\noindent \textbf{Semester Schedule}
\begin{small}
\vspace{5mm}

\scalebox{.82}{\hspace{-.2cm}\begin{tabular}{ L{2.2cm}L{1.4cm}L{7.6cm}L{3.5cm}C{3.1cm}}
\hline \hline
 Date & Day   & Topic
     & Pages in Notes & Homework Due  \\
\midrule
\textbf{Week 1}   \\
Aug 30   &   T    &  Ch1-2: Introduction to Data Science  &   1-6 &       \\
Sep 1   &   Tr    &  Ch3: Variables and Data   &   7-17   &    \\
\textbf{Week 2}   \\
Sep 6  &   T    &  ChX: Introduction to R and RMarkdown &         \\
Sep 8   &   Tr    &  Ch3: Visualization   &   30-42 &  \#1  \\
\textbf{Week 3}   \\
Sep 13  &   T    &  Ch3: Visualization   &   43-57 &       \\
Sep 15   &   Tr    &  Ch5: Data Transformations   &   58-64  &  \#2   \\
\textbf{Week 4}   \\
Sep 20  &   T    &  Ch6: Introduction to Data Wrangling   &   65-73 &       \\
Sep 22   &   Tr    &  Ch6: Introduction to Data Wrangling  &   74-78 & \#3  \\
\textbf{Week 5}   \\
Sep 27  &   T    &  Ch6: Introduction to Data Wrangling     & 78-88  &   \\
Sep 29   &   Tr    &  Ch7: Unsupervised Learning &   88-96 &  \#4  \\
\textbf{Week 6}   \\
Oct 4 &   T    &  Ch7: Dissimilarity Measurements  &  97-106  &   \\
Oct 6   &   Tr    &  Ch7: Inter-group Proximity Measures  &   106-117 &  \#5 \\
\textbf{Week 7}   \\
Oct 11 &   T    &  Ch7: Dendograms  &  118-126  &      \\
Oct 13   &   Tr   & \textbf{Midterm \#1}   &   &      \\
\textbf{Week 8}   \\
Oct 18 &   T    &  \textit{Fall Break: No Class}  &    &      \\
Oct 20   &  Tr   &  Ch7: Standardizing Variables  &  127-136  &     \#7 \\
\textbf{Week 9}   \\
Oct 25 &   T   &  Ch7: K-means Clustering  &   137-144 &    \\
Oct 27   &   Tr    &  Ch7: Between Group Sum of Squares   & 145-151  &   \#8    \\
\textbf{Week 10}   \\
Nov 1 &   T    &  Ch7: Similarity Measures for Binary Data  &   152-163 &    \\
Nov 3   &   Tr    &  ChX: Text Analysis &  &  \#9 \\
\textbf{Week 11}   \\
Nov 8 &   T    & ChX: Text Analysis &   &    \\
Nov 10   &   Tr    &  Ch8: Supervised Learning: Classification/Pred. &   185-193 &   \#10 \\
\textbf{Week 12}   \\
Nov 15 &   T    & Ch10: Supervised Learning: Linear Regression &   193-199 &     \\
Nov 17   &   Tr    &  Ch10: Supervised Learning: Decision Tree &   200-211 &  \#11   \\
\textbf{Week 13}   \\
Nov 22 &   T    &  Ch10: Supervised Learning: Decision Tree, app &   212-219 &    \\
Nov 24   &   Tr    & \textit{Thanksgiving Break: No Class} &    &       \\
\textbf{Week 14}   \\
Nov 29 &   T    &  Ch10: Supervised Learning: knn reg &  220-227  & \\
Dec 1   &   Tr    &  \textbf{Midterm \#2}  &  228-235  &      \\
\textbf{Week 15}   \\
Dec 6 &   T    & Ch10: Probability: Logistic Regression &    &    \\
Dec 8   &   Tr    & Ch10: Probability: knn classification &   & \#12   \\
& \\
Dec 12    &   M    &  \textbf{Final Project, due by 5:00pm}  & &  \\


\bottomrule
\end{tabular}}


\end{small}
\fi

\end{document}
